\chapter{Introduction}

\section{Motivation}
As part of my bachelor degree course in software-systems I had to complete a three-month informatics internship at an organization of my choice. I did the internship in summer 2015 at General Acoustics e.K.\footnote{\label{foot:1} www.generalacoustics.com} located in Kiel, Germany. General Acoustics e.K. is specialized at producing leading edge echo-sounders, water level-, and wave sensors as well as flow measuring systems. They are primarily using remote sensing technologies especially ultrasound based technologies. Apart from the hardware General Acoustics e.K. also provides specialized software to provide a complete customer experience.\\
In these three month of internship I helped in the development of a database solution to store, process, and visualize environmental sensor data from an offshore platform located in the Iraq. The internship was very successful for both sides and, thus resulting in my further involvement in their work. The good cooperation with General Acoustics e.K. made it possible to do this software project.%notsogood TODO\\ 

The most complex sensor on the Iraqi platform is an Acoustic Doppler Current Profiler (ADCP), which measures the flow direction and velocity of the water at different heights. It operates at around 6Hz and produces a lot of data, a big part of it currently not important for our analysis. In the present situation all data gets logged on the platform, and is later transmitted onshore. On the onshore server the incoming data is parsed, processed, and then stored in a database. Problems rise due to the unreliable communication between platform and onshore server caused by regular power outages. The amount of data produced by the ADCP is too high to be logged over a long time and lead in the past to data loss.\\\newline
To prevent such inconveniences in future projects General Acoustics e.K. decided to move some of the data-processing activities onto the offshore platform, for example the parsing of the data and only keeping what is necessary, reduces the needed logging space significantly. Besides the experience in current projects, the decision was also driven by possible upcoming projects where multiple platforms will be managed from a central station. This move implicated that the logging device has to be more powerful, it should be able to parse and process the incoming data in real time as well as being able to run other services like a small database. In the future the the logging device on the platform should be able to offer a stand-alone experience of the whole measuring platform instead of just log the data.\\\newline
The goal of this software project is the design and implementation of a modular parser for ADCP data. Modularity in this context is meant as a component based, reusable code implementation. The resulting software will be the foundation of the concept above, it requires easy adaptability to support other sensors, data processing, and data analysis methods. 

\section{Description of Work}
The work of this project can be divided into two parts. The first part consisted of knowledge gathering regarding the ADCP technology as well as refinement of the requirements specified from General Acoustics e.K. These requirements define in the end the tools and libraries used later on in the implementation. Based on these requirements the design of the application was decided. A determining factor was for example the requirement of the programming language, the project should be written in platform independent C++.

The second part was the implementation of the components, first the standalone components without dependencies, they should be usable outside this project in other contexts, and later the main program, which integrates the previous components. The software had to compile and run on an x64 architecture with Windows and an arm-hf architecture with Debian Linux.

%The last part was a test and bug-fixing phase where the software was tested under various conditions and scenarios. A key factor was the robustness under real-time operation.  %TODO expand
\section{Thesis Outline}
The report for this project is structured as follows: First, the related works are highlighted in Chapter 2 where the previous work from General Acoustics is shown, followed by an introduction to the ADCP technology and an overview of the used libraries is presented. Following up on the related works in Chapter 3, the design decision will be presented. The decisions both architectural and implementation specific are reasoned based on the requirements from General Acoustics e.K..\\ 
In Chapter 4, the implementation of the components will be discussed in detail. The report ends with Chapters 5, in which the evaluation of the implementation is discussed, before the report is concluded in Chapter 6.
